\graphicspath{{images/}}

\section{\thesection~Methods}
\label{sec:methods}

\subsection{\thesubsection~CANS}

To analyse QFA data using the competition model I developed the Python
package CANS which can be used for model composition, model
simulation, parameter inference, and visualisation of results. CANS
accepts cell density timecourses for any size rectangular array. CANS
can produce SBML models to document results of parameter inference or
for independent validation using other simulation tools. It is
relatively simple to create and simulate new models involving
reactions between species within cultures or between neighbouring
cultures, and to fit these provided an initial guess. The CANS package
is available at
\href{https://github.com/lwlss/CANS}{https://github.com/lwlss/CANS}.

\subsection{\thesubsection~Solving and fitting}

\subsubsection{\thesubsubsection~Solving}
\label{sec:solving_comp}
% [Only keep if I remove the CANS overview above: To analyse QFA data I
% created a Python package (``CANS'') for model composition, model
% simulation, parameter inference, and visualisation of results. CANS
% accepts timecourse data of cell density for any size rectangulare
% array.]\\

CANS numerically solves models using one of two methods. The first is
slower and uses SciPy's integrate.odeint to solve models written in
Python at user supplied timepoints. I vectorised code using NumPy to
optimise solving of the competition model by this method. For solving
a plate of 384 cultures with cell density observations at 10 unevenly
spaced time points, I found that using the Python bindings for
libRoadRunner was about 10 times faster. libRoadRunner requires models
to be written in SBML so I wrote code using the libSBML Python API to
automatically generate SBML versions of the competition model for any
size plate.
% //I could go into more detail about how models are defined// but
% maybe this is not needed.
Unlike SciPy's odeint, libRoadRunner only simulates at uniformly
spaced timepoints. To fit QFA cell observations, which are not made at
fixed time intervals, requires simulated cell amounts at the observed
timepoints. For the analysis in (P15 section), where each timecourse
has only 10 timepoints, I simulated sequentially between (pairs of)
timepoints. This method was slower for the analysis in (Stripes
section) where each timecourse had around 50 timepoints. To increase
speed I used SciPy's interpolate.splrep to make a 5th order B-spline
of cell density timecourses with smoothing condition \(s=1.0\). I
evaluated the spline for cell density using SciPy's interpolate.splev
at 15 evenly spaced intervals from time zero to the time of the last
QFA observation. I then solved these
timecourses with one call to RoadRunner.simulate.\\\\

% Fitting
\subsubsection{\thesubsubsection~Fitting the competition model}
\label{sec:fitting_comp}

I use QFA data after processing with Colonyzer \citep{Lawless2010}.
Colonyzer uses integrated optical density measurements in whole plate
images as a proxy for cell density. I used timecourse cell density
estimates, which have arbitrary units, throughout my analysis. I fit
the competition model using a gradient method and made maximum
likelihood estimates of parameters using a normal model of measurement
error. For constrained minimisation I used the L-BFGS-B algorithm from
SciPy's integrate package.

I determined stopping criteria so that parameters of full-plate
simulated data sets, with a small amount of simulated noise, were
recovered with high precision. To help the minimizer, I scaled
\(C(0)\) values by a factor of \(10^{5}\) to make parameter
values closer in order of magnitude. I ran repeated fits using
different parameter guesses for each plate (see Section~(P15 and
Stripes details)). I set bounds according to
Table~\ref{tab:p15_bounds} and checked that best fits had no
parameters at a boundary.
%
\columnbreak
\begin{center}
  \captionof{table}{\textbf{Parameter bounds.} Used for fitting the
    competition model to P15 and the Stripes and Filled plates. Bounds
    on \(N(0)\) were applied to both \(N^{I}(0)\) and
    \(N^{E}(0)\) for internal and edge cultures. ``guess'' refers
    to the initial guess (see Section~\ref{sec:initial_guess}).}
  \begin{tabular}{| c | c c |}
    \hline
    Parameter        & Lower Bound  & Upper Bound \\
    \hline
    \(C(0)\)     & guess x \(10^{-3}\)  & guess x \(10^{3}\)\\
    \(N(0)\)     & guess / \(2\)  & guess x \(2\)\\
    % \(N^{I}(0)\) \& \(N^{E}(0)\) & guess / \(2\)  & guess x \(2\)\\\\
    \(k\)        & 0.0    & 10.0\\
    % \(b\) (all cultures)           & 0.0    & \(\infty\) \\
    \(b\)           & 0.0    & None \\
    \hline
  \end{tabular}
  \label{tab:p15_bounds}
\end{center}
%
%%%% Boundary conditions Two N_0 %%%%%
Cultures at the edge of a plate have an advantage because they have
access to a greater area of nutrients. I corrected for this using a
separate parameter \(N^{E}(0)\) representing a higher initial
amount of nutrients in edge cultures. In rate equations involving edge
cultures, I scaled edge culture nutrient amount \(N_{i}\) by the ratio
\(N^{I}(0)/N^{E}(0)\), where \(N^{I}(0)\) is the amount
of nutrients in internal cultures. The physical interpretation of this
correction is that edge cultures have an extra supply of nutrients
that can diffuse instantly into the reaction volume. This treatment
reduced the error in cell density estimates for cultures one row or
column inside the edge and resulted in better fits to internal
cultures overall (see Table~\ref{tab:corner} or Section). Cell density
measurements from edge cultures contain more noise due to reflections
from plate walls \citep{Lawless2010}. I collectively fit to all
cultures and selected best fits based on only the fit to internal
cultures.

%%%% Empties %%%%%
(Can go to results section or Stripes method section:) QFA data for
the Stripes plate contained observations for cultures that were known
to be empty. When fitting the competition model, I set growth constant
\(b\) to zero for these cultures and removed them from fitting.
%%%% End Empties %%%%%

\subsubsection{\thesubsubsection~Fitting the logistic model}

Fitting the mass action logistic model requires using culture level
\(N(0)\) and creating 383 extra parameters. The QFA R package
\citep{qfa2016} can fit the standard logistic model and has heuristic
checks to correct a confounding of parameters that occurs when
slow-growing cultures are dominated by noise. I did not have time to
implement these checks for the mass action logistic model, so I
instead fit the standard logistic model using the QFA R package. This
is not equivalent because QFA R does not fit data collectively and
instead uses a culture level \(C(0)\). However, this is a useful
comparison with a method of analysis currently used in QFA (see
e.g. \citet{Addinall2011}). I do not expect much disagreement of
fitness estimates with the mass action logistic model once heuristic
checks are implemented. In contrast to the competition model, noisy
data from edge cultures was discarded before fitting. I conduct model
comparison between the competition and logistic models in sections
(Results sections).

\subsubsection{\thesubsubsection~Data visualisation}

I created plotting functions in CANS to visualise fits and simulations
of QFA timecourses and to compare the ranking of fitness estimates
using the Python package matplotlib.

\subsection{\thesubsection~Parameter conversion}
\label{sec:parameter_conversion}

When \(k\) is set to zero, the competition model
(\ref{eq:competition_model}) reduces to the mass action logistic model
which has the same sigmoidal solution as the standard logistic
model. In this limit, it is possible to equate C species of both
models and convert parameters using,
% Derivation or link to blog.
\begin{subequations}
  \label{eq:conversion}
  \begin{align}
    &r_{i} = b_{i}(C(0) + N(0)),\\
    &K = C(0) + N(0).
    % &r = b(C(0) + N(0))\\
    % &K = (C(0) + N(0))
  \end{align}
\end{subequations}
%
The reaction equation of the competition model (\ref{eq:reaction})
assumes that all nutrients are converted to cells. This implies that
all cultures starting with the same amount of nutrients reach the same
final amount of cells. Therefore, to fit the mass action logistic
model to QFA data, it is necessary to allow \(N(0)\) to vary for each
culture which is not physically realistic and, in which case, the mass
action logistic model has the same number of parameters (769) as the
standard logistic model.
%
Figure~\ref{fig:correction} shows fits of a single culture on a larger
16x24 format plate using both models. This culture grew faster than
its neighbours (not shown) and, according to the competition model,
competed for more nutrients.
%
Figure~\ref{fig:correction}a shows the mass-action logistic model fit
where \(N(0)\) is estimated as being approximately equal to the
final cell amount, or carrying capacity \(K\).
%
Figure~\ref{fig:correction}b shows the competition model fit with a
plate level \(N(0)\) and \(k > 0\). Re-simulating with \(k\)
set to zero gives the dashed mass action logistic model curves which
are corrected for competition. We can therefore obtain the corrected
logistic model \(r_{i}\) and \(K_{i}\) of these curves by converting
from competition model estimates of \(b_{i}\), \(C(0)\), and
\(N(0)\). N.B. \(b\) is the same for both the solid and dashed
curves in Figure~\ref{fig:correction}b.

% This produces the correction in \(r\) and \(K\) between the two fits
% (see Figures~\ref{fig:correction}) and allows direct comparison
% between competition and logistic model estimates.
%

Competition model \(C(0)\) and \(N(0)\) are the same for all
cultures on a plate. Therefore, by the conversion equations
(\ref{eq:conversion}), all cultures on a plate have the same carrying
capacity \(K\) and all \(b_{i} \propto r_{i}\) by the same
factor. Similarly, \(MDP\) is the same for all cultures and all
\(b_{i} \propto MDR_{i}\) by the same factor (see
Equation~\ref{eq:MDR_MDP}). Therefore, \(b\) is equivalent to all
common QFA fitness measures, \(r\), \(MDR\), and \(MDR*MDP\) (see
e.g. \citet{Addinall2011} and \citet{qfa2016}). This makes \(b\) a
very convenient fitness measure for the competition model; we need not
convert to logistic model parameters to compare the fitness rankings
of cultures on the same plate. To compare competition model fitness
rankings between different plates we can of course use \(b\). However,
this is not equivalent to comparing \(r\) or \(MDR\) as different
plates may have different \(C(0)\) and \(N(0)\).

\begin{Figure}
  \centering
  \graphicspath{{images/correction/}}
  \includegraphics[width=\linewidth]{final/logistic}
  \includegraphics[width=\linewidth]{final/competition}
  \captionof{figure}{\textbf{Using the competition model to correct
      for competition.} Fits are to culture (R10, C3) of P15 which
    grew faster and reached a higher final cell density than its
    neighbours (not shown). According to the competition model, this
    is because this culture competed for more nutrients. To reach the
    same final cell density, the logistic equivalent model requires a
    higher amount of starting nutrients for this culture and a
    different amount for each neighbour. The correction to the
    competition model simulates how growth would have appeared without
    competition and, for the culture shown,
    gives~\(r_{corrected} > r_{logistic}\)~and~\(K_{corrected} < K_{logistic}\).}
  \label{fig:correction}
\end{Figure}

\subsection{\thesubsection~Determining initial parameters}
\label{sec:initial_guess}

Fitting the competition model to a standard 384-format QFA plate is a
formidable optimisation problem involving 384 timecourses and 387
parameters. Achieving good fits therefore requires making a good
initial guess. To fit small simulated zones I could simply use many
random parameter guesses. However, for a full plate the chance of any
random guess being close to the ``true'' values diminishes and more
sophisticated guessing methods are required. I developed the
\textit{Imaginary Neighbour Model} (Figure~(ref)) for guessing
competition model \(b\) and this allowed good fits of to be made. As
the project progressed, it became clear how to convert between fast
logistic parameter estimates and competition model parameters (see
Section~\ref{sec:parameter_conversion}) and this could be a valid
alternative.

\subsubsection{\thesubsubsection~Guessing initial amounts}
\label{sec:guessing_amounts}

Recall from the competition model reaction equations
(\ref{eq:reaction} and \ref{eq:diffusion_reaction}) that nutrients can
only diffuse or be converted to cells. Thus, assuming that reactions
are nearly complete at the end of cell observations and that
\(C(0) \ll C(\infty)\), the total initial amount of nutrients,
\(N_{Tot}\), can be estimated using,
\begin{equation}
  \label{eq:N_Tot}
  N_{Tot} = n_{I}N_{I}(0) + n_{E}N_{E}(0) \approx C_{F},
\end{equation}
where \(C_{F}\) is the total of final cell measurements, \(n_{I}\) and
\(n_{E}\) are the numbers of internal and edge cultures, and
\(N_{I}(0)\) and \(N_{E}(0)\) are initial nutrient amounts
for internal and edge cultures (see
Section~\ref{sec:fitting_comp}). Using (\ref{eq:N_Tot}) and an estimate
for the ratio of area associated with edge cultures to area associated
with internal cultures,
\(A_{r} = A_{E} / A_{I} = N_{E}(0) / N_{I}(0)\), I made
guesses of \(N_{I}(0)\) and \(N_{E}(0)\) using,
%
\begin{equation}
  \label{eq:N_0_guesses}
  \begin{aligned}
    % &A_{r} = A_{E} / A_{I} = N_{E}(0) / N_{I}(0)\\
    % &N_{Tot} = n_{I}N_{I}(0) + n_{E}N_{E}(0) \approx C_{F}\\
    &N_{I}(0) = N_{Tot} / (n_{I} + n_{E}A_{r})\\
    &N_{E}(0) = N_{Tot} / (n_{I}/A_{r} + n_{E}).
    % &N_{E}(0) = (N_{Tot} - n_{I}N_{I}(0)) / n_{E}.
  \end{aligned}
\end{equation}
%
When \(A_{r} = 1\), (\ref{eq:N_0_guesses}) reduces to the initial
nutrient guess for the one initial nutrient parameter model. I used
\(A_{r} = 1.4\).
\\
In QFA using dilute cultures, \(C(0)\) falls below the level of
detection. I did not estimate initial guesses of \(C(0)\) and instead
ran multiple fits over a range of \(C(0)\) values in logspace chosen
to encompass uncertainty in \(C(0)\) for the given experiment.

% ; for P15 \(10^{-5}\) to \(10^{-3}\) times the
% average final cell amount, for the Stripes and Filled plates
% \(10^{-7}\) to \(10^{-1}\) times average final cell amount.


\subsubsection{\boldmath \thesubsubsection~Guessing \(b\) \unboldmath}
%%%%%%%%%%%%%%% Imag neigh %%%%%%%%%%%%%%%%%%%

To guess competition model \(b_{i}\) I used the imaginary neighbour
model (Figure~(ref)) to quickly fit individual cultures. The model is
based on the reaction and rate equations of the competition
model~(\ref{eq:reaction}--\ref{eq:diffusion_reaction}) but tries to
replicate the diffusion of nutrients into and out of a culture using
imaginary fast and slow growing neighbours with different nutrient
diffusion constants \(k_{f}\) and \(k_{s}\). The growth constants of
the fast and slow growing cultures are \(b_{f}\) and \(b_{s}\). A
schematic of the model is drawn in . To fit the model to QFA data, I
fixed \(C(0)\) and \(N_{I}(0)\) for all cultures by the initial
guesses (see Section~\ref{sec:guessing_amounts}); I fixed \(b_{f}\) at
a range of different guesses, and fixed \(b_{s} = 0\); I allowed
\(b\), \(k_{f}\), and \(k_{s}\) to vary. I determined the number,
\(n\), of each neighbour from the guess of \(N_{I}(0)\) and the range
of final cell amounts, such that the culture with the highest observed
final cell density had enough slow growing neighbours to provide all
of the nutrients necessary to reach this final cell density. I solved
the imaginary neighbour model using SciPy's odeint. I fit using a
gradient method as in Section~\ref{sec:fitting_comp}. Fits of the
imaginary neighbour model took several minutes which is much faster
than fits of the full competition model [(will probably have mentioned
time above:) which took several hours].

%%%%%%%%%%%%%%% Imag neigh %%%%%%%%%%%%%%%%%%%%%%%%%%%%%%%%%%%%%%

\subsubsection{\boldmath \thesubsubsection~Guessing \({k}\) \unboldmath}

Simulations of the competition model using sets of \(b\) parameters
drawn from different normal distributions have linear relationships
between variance in final cell amount and nutrient diffusion constant
\(k\). I simulated guessed parameters \(C(0)\), \(N(0)\), and
\(b_{i}\) with a range of different \(k\) values and used linear
regression to parameterise the straight line. I then took the variance
in final cell amount for real data and guessed \(k\) from the
straight line.
% \\\\
% Don't think I need this figure. r->b and make titles bigger. Fitted
% \(k\) was much higher so I should probably resimulate over a
% bigger range. Only possibly to know in hindsight. I could also use the
% fitted parameters for this rather than drawing from a normal
% distribution.
% \\
% \graphicspath{{images/guessing/}}
% \begin{Figure}
%   \centering
%   \includegraphics[width=\linewidth]{final/kn_guessing}
%   \captionof{figure}{\textbf{Guessing \(\bm{k}\) from the variance in
%       final cell amounts.} The competition model is simulated for a
%     16x24 format plate using two random sets of culture-level \(b\)
%     parameters drawn from different normal distributions. Each set of
%     \(b\) parameters is simulated with a range of \(k_n\) parameter
%     values. The variance in final cell density for all cultures is
%     plotted against \(k_n\) for each simulation. Lines are shown for
%     least squares fits to points from each set of \(b\) parameters.}
%   \label{fig:kn_guessing}
% \end{Figure}

\subsection{\thesubsection~Development of a genetic algorithm}

% I began work on a hierarchical genetic algorithm method of parameter
% inference inspired by the hierarchical Bayesian analysis of QFA data
% by \citet{Heydari2016}. (I'm not sure that it is too similar or was in
% fact inspired by this.)

\subsection{\thesubsection~Model comparison using a single QFA plate}
% cell_ratios = np.logspace(-3, -5, num=5)

\subsection{\thesubsection~Cross-plate calibration and validation}
% cell_ratios = np.logspace(-1, -7, num=10) I increased the range to
% account for higher inoculum density and work of Herrmann.
%
% Results \(C(0)\), \(N_{I}(0)\), \(N_{E}(0)\), \(k\)
% est_params Stripes [ 0.00831517,  0.08524787,  0.09564223,  1.92525936]
% est_params Filled [ 0.00617039,  0.11660628,  0.1830217 ,  4.84354755]


%%% Local Variables:
%%% mode: latex
%%% TeX-master: "report"
%%% End:
