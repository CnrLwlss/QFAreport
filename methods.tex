\graphicspath{{images/}}

\section{\thesection~Methods}
\label{sec:methods}

To analyse QFA data I created a Python package (``CANS'') for model
composition, model simulation, parameter inference, and visualisation
of results. It accepts timecourse data from QFA using any size
rectangular array. CANS can produce SBML models to document fitting
results which can be validated independently using a variety of
tools. It is relatively simple to create new models, similar in form
to the competition model, containing reactions between species within
cultures or between neighbouring cultures. These can also be simulated
and fit. (Methods for making initial guesses are model specific.) The
CANS package is available for download at (github).

\subsection{\thesubsection~Tools, solving, and fitting}

To analyse QFA data I created a Python package (``CANS'') for model
composition, model simulation, parameter inference, and visualisation
of results. It accepts timecourse data of culture density for any size
rectangular array. I use QFA data after processing with Colonyzer
\citep{Lawless2010} to produce timecourses of cell density for each
culture on a plate. Colonyzer processes series of time-stamped whole
plate images and uses integrated optical density measurements as a
proxy for cell density. I use these cell density estimates, with
arbitrary units, throughout my analysis.



\subsection{\thesubsection~Parameter conversion}

(Could move to discussion: The identity of the
nutrient molecule is unknown and it is not clear whether metabolism of
the nutrient molecule will have a significant effect. If necessary a
metabolism reaction could also be modelled.)\\

When \(k_{n}\) is set to zero, the competition model
(\ref{eq:competition_model}) reduces to the mass action logistic model
which and has the same sigmoidal solution as the standard logistic
model. In this limit, it is possible to equate cells of both models
and convert parameters using (\ref{eq:conversion}) (see Conor's blog
for a derivation).
% Derivation or link to blog.
\begin{subequations}
  \label{eq:conversion}
  \begin{align}
    &r_{i} = b_{i}(C_{t_{0}} + N_{t_{0}})\\
    &K = (C_{t_{0}} + N_{t_{0}})
    % &r = b(C_{t_{0}} + N_{t_{0}})\\
    % &K = (C_{t_{0}} + N_{t_{0}})
  \end{align}
\end{subequations}
%
The reaction equation of the competition model (\ref{eq:reaction})
assumes that all nutrients are converted to cells. This implies that
cultures starting with the same amount of nutrients reach the same
final amount of cells. To fit the mass action logistic model, it is
necessary to allow \(N_{t_{0}}\) to vary for each culture which is not
physical. In this case, the mass action logistic model has the same
number of parameters (769) as the standard logistic model. (Probably
repetition: When I fit the competition model I collectively fit the
timecourses of all cultures on a plate using a plate level
\(N_{t_{0}}\) and 387 parameters.)
%
Figure~\ref{fig:correction} shows fits of a single culture on a larger
16x24 format plate using both models. This culture grew faster than
its neighbours (not shown) and, according to the competition model,
competed for more nutrients.
%
Figure~\ref{fig:correction}a shows the mass-action logistic model fit
where \(N_{t_{0},i}\) is estimated as being approximately equal to the
final cell amount, or carrying capacity \(K_{i}\).
%
Figure~\ref{fig:correction}b shows the competition model fit with a
plate level \(N_{t_{0}}\) and \(kn > 0\). Re-simulating with \(k_{n}\)
set to zero to gives the dashed curves that are equivalent to the
logistic model and which we would be observe if there were no
competition. Competition can therefore be corrected for by converting
competition model estimates of \(b_{i}\), \(C_{t_{0}}\), and
\(N_{t_{0}}\) to independent model \(r_{i}\) and \(K_{i}\) by
(\ref{eq:conversion}).
% This produces the correction in \(r\) and \(K\) between the two fits
% (see Figures~\ref{fig:correction}) and allows direct comparison
% between competition and logistic model estimates.
%

In the competition model, \(C_{t_{0}}\) and \(N_{t_{0}}\) are the same
for all cultures on a plate. Therefore, by the conversion equations
(\ref{eq:conversion}), all cultures have the same carrying capacity
\(K\) and all \(b_{i} \propto r_{i}\) by the same factor. Similarly
all \(b_{i} \propto MDR_{i}\) by the same factor and \(MDP\) is the
same for all cultures (see equation~\ref{eq:MDR_MDP}). \(b_{i}\) is
therefore equivalent to the QFA fitness estimates currently in use
(see e.g. \citet{Addinall2011} and \citet{qfa2016}). This makes
\(b_{i}\) an ideal fitness estimate for the competition model; we need
not convert to logistic model parameters to compare the fitness
rankings of cultures on the same plate. To compare competition model
fitness rankings between different plates we can of course use
\(b_{i}\). However, this is not equivalent to comparing \(r_{i}\) or
\(MDR_{i}\) as different plates may have different \(C_{t_{0}}\) and
\(N_{t_{0}}\).

\begin{Figure}
  \centering
  \graphicspath{{images/correction/}}
  \includegraphics[width=\linewidth]{final/logistic}
  \includegraphics[width=\linewidth]{final/competition}
  \captionof{figure}{\textbf{Using the competition model to correct
      for competition.} Fits are to culture (R10, C3) of P15 which
    grew faster and reached a higher final cell density than its
    neighbours (not shown). According to the competition model, this
    is because this culture competed for more nutrients. To reach the
    same final cell density, the logistic equivalent model requires a
    higher amount of starting nutrients for this culture and a
    different amount for each neighbour. The correction to the
    competition model simulates how growth would have appeared without
    competition and allows us to return parameters \(r\) and \(K\) of
    the logistic model.}
  \label{fig:correction}
\end{Figure}


%%% Local Variables:
%%% mode: latex
%%% TeX-master: "report"
%%% End:

\subsection{\thesubsection~Making an initial guess}
\subsection{\thesubsection~Development of a genetic algorithm}
\subsection{\thesubsection~Model comparison using a single QFA plate}
\subsection{\thesubsection~Cross-plate calibration and validation}
