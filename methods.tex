\graphicspath{{images/}}

\section{\thesection~Methods}
\label{sec:methods}

To analyse QFA data I created a Python package (``CANS'') for model
composition, model simulation, parameter inference, and visualisation
of results. It accepts cell density timecourses for any size
rectangular array. CANS can produce SBML models to document results of
parameter inference or for independent validation. It is relatively
simple to create and simulate new models, if they involve reactions
between species within cultures or between neighbouring cultures, and
to fit these provided an initial guess. The CANS package, containing
source code for the results in this paper, is available at
\href{https://github.com/lwlss/CANS}{https://github.com/lwlss/CANS}.

\subsection{\thesubsection~Tools, solving, and fitting}

% [Only keep if I remove the CANS overview above: To analyse QFA data I
% created a Python package (``CANS'') for model composition, model
% simulation, parameter inference, and visualisation of results. CANS
% accepts timecourse data of cell density for any size rectangulare
% array.]\\

I use CANS to parse QFA data after processing with Colonyzer
\citep{Lawless2010}. Colonyzer processes series of time-stamped
whole-plate images, using integrated optical density measurement as a
proxy for cell density, to produce timecourses of cell density for all
cultures on a plate. I use these cell density estimates, which have
arbitrary units, throughout my analysis. CANS numerically solves
models using one of two methods. The first is slower and uses SciPy's
integrate.odeint to solve models written in Python at user supplied
timepoints. I vectorised code using NumPy to optimise solving of the
competition model by this method. For solving a plate of 384 cultures
with cell density observations at 10 unevenly spaced time points, I
found an approximately 10 times further increase in speed using the
Python bindings for the package libRoadRunner. libRoadRoadrunner's
RoadRunner.simulate requires models to be written in SBML so I wrote
code using the libSBML Python API to automatically generate SBML
versions of the competition model for any size plate.

% //I could go into more detail about how models are defined// but
% maybe this is not needed.

Unlike SciPy's odeint, libRoadRunner can only simulate at uniformly
spaced timepoints. To fit QFA timecourses, which do not have fixed
intervals, requires simulated cell amounts at the observed
timepoints. For the analysis in (P15 section), where each timecourse
has only 10 timepoints, I simulated using a function call between each
pair of adjacent timepoints. This method was slower for the analysis
in (Stripes section) where each timecourse had around 50 timepoints. To
increase speed I used SciPy's interpolate.splrep to make a 5th order
B-spline of cell density timecourses with smoothing condition
\(s=1.0\). I evaluated the spline for cell density using SciPy's
interpolate.splev at 15 evenly spaced intervals from time zero to the
time of the last QFA observation. I then solved these timecourses with
one call to RoadRunner.simulate.

QFA data for the Stripes plate (sec ref) contained observations for
cultures that were known to be empty. When fitting the competition
model (\ref{eq:competition_model}), I set growth constant \(b\) to
zero for these cultures and removed them from the objective function.


% tolerances - These tolerances returned estimated parameters with a
% high precision when using simulated data sets.

%initial amounts
%time points and splining

% Displayed output using matplotlib

% Automatic SBML model composition
% Should I include a code example of defining a model?

% SolvingThis uses libRoadRunners

\subsection{\thesubsection~Parameter conversion}

(Could move to discussion: The identity of the
nutrient molecule is unknown and it is not clear whether metabolism of
the nutrient molecule will have a significant effect. If necessary a
metabolism reaction could also be modelled.)\\

When \(k_{n}\) is set to zero, the competition model
(\ref{eq:competition_model}) reduces to the mass action logistic model
which has the same sigmoidal solution as the standard logistic
model. In this limit, it is possible to equate cells of both models
and convert parameters using (\ref{eq:conversion}) (see Conor's blog
for a derivation).
% Derivation or link to blog.
\begin{subequations}
  \label{eq:conversion}
  \begin{align}
    &r_{i} = b_{i}(C_{t_{0}} + N_{t_{0}})\\
    &K = (C_{t_{0}} + N_{t_{0}})
    % &r = b(C_{t_{0}} + N_{t_{0}})\\
    % &K = (C_{t_{0}} + N_{t_{0}})
  \end{align}
\end{subequations}
%
The reaction equation of the competition model (\ref{eq:reaction})
assumes that all nutrients are converted to cells. This implies that
all cultures starting with the same amount of nutrients reach the same
final amount of cells. Therefore, to fit the mass action logistic
model to QFA data, it is necessary to allow \(N_{t_{0}}\) to vary for
each culture which is not physical and, in which case, the mass action
logistic model has the same number of parameters (769) as the standard
logistic model. (Probably repetition: When I fit the competition model
I collectively fit the timecourses of all cultures on a plate using a
plate level \(N_{t_{0}}\) and 387 parameters.)
%
Figure~\ref{fig:correction} shows fits of a single culture on a larger
16x24 format plate using both models. This culture grew faster than
its neighbours (not shown) and, according to the competition model,
competed for more nutrients.
%
Figure~\ref{fig:correction}a shows the mass-action logistic model fit
where \(N_{t_{0}}\) is estimated as being approximately equal to the
final cell amount, or carrying capacity \(K\).
%
Figure~\ref{fig:correction}b shows the competition model fit with a
plate level \(N_{t_{0}}\) and \(kn > 0\). Re-simulating with \(k_{n}\)
set to zero gives the dashed mass action logistic model curves which
are corrected for competition. We can therefore obtain the corrected
logistic model \(r_{i}\) and \(K_{i}\) of these curves by converting
from competition model estimates of \(b_{i}\), \(C_{t_{0}}\), and
\(N_{t_{0}}\). N.B. \(b\) is the same for both the solid and dashed
curves in Figure~\ref{fig:correction}b.

% This produces the correction in \(r\) and \(K\) between the two fits
% (see Figures~\ref{fig:correction}) and allows direct comparison
% between competition and logistic model estimates.
%

Competition model \(C_{t_{0}}\) and \(N_{t_{0}}\) are the same for all
cultures on a plate. Therefore, by the conversion equations
(\ref{eq:conversion}), all cultures on a plate have the same carrying
capacity \(K\) and all \(b_{i} \propto r_{i}\) by the same
factor. Similarly, \(MDP\) is the same for all cultures and all
\(b_{i} \propto MDR_{i}\) by the same factor (see
Equation~\ref{eq:MDR_MDP}). Therefore, \(b\) is equivalent to all
common QFA fitness measures, \(r\), \(MDR\), and \(MDR*MDP\) (see
e.g. \citet{Addinall2011} and \citet{qfa2016}). This makes \(b\) a
very convenient fitness measure for the competition model; we need not
convert to logistic model parameters to compare the fitness rankings
of cultures on the same plate. To compare competition model fitness
rankings between different plates we can of course use \(b\). However,
this is not equivalent to comparing \(r\) or \(MDR\) as different
plates may have different \(C_{t_{0}}\) and \(N_{t_{0}}\).

\begin{Figure}
  \centering
  \graphicspath{{images/correction/}}
  \includegraphics[width=\linewidth]{final/logistic}
  \includegraphics[width=\linewidth]{final/competition}
  \captionof{figure}{\textbf{Using the competition model to correct
      for competition.} Fits are to culture (R10, C3) of P15 which
    grew faster and reached a higher final cell density than its
    neighbours (not shown). According to the competition model, this
    is because this culture competed for more nutrients. To reach the
    same final cell density, the logistic equivalent model requires a
    higher amount of starting nutrients for this culture and a
    different amount for each neighbour. The correction to the
    competition model simulates how growth would have appeared without
    competition and allows us to return parameters \(r\) and \(K\) of
    the logistic model.}
  \label{fig:correction}
\end{Figure}


%%% Local Variables:
%%% mode: latex
%%% TeX-master: "report"
%%% End:

\subsection{\thesubsection~Making an initial guess}
\subsection{\thesubsection~Development of a genetic algorithm}
\subsection{\thesubsection~Model comparison using a single QFA plate}
\subsection{\thesubsection~Cross-plate calibration and validation}
