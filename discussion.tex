\graphicspath{{images/}}

\section{\thesection~Discussion}
\label{sec:discussion}

Fitness ranking from competition model fits may be better than from
logistic model fits (Will comparing stripes rankings reveal
anythin?). However, we cannot quantitatively compare fitness estimates
between plates because we are not finding global minima. Work has
begun to develop a genetic algorithm to do so. I am not convinced that
this will succeed because growth is systematically overestimated when
we move from the filled to striped plate for all of the current best
parameter solutions. This suggests an issue with the modelling
approach; below I suggest ways in which this could be improved. In
any case, qualitative cross-plate validation using order of fitness
ranking may still be better (for the competition model).

The first thing to notice about QFA data - from P15, the striped
plate, and the filled plate - is the characteristic endpoint in growth
on each plate. This holds whether a region contains many fast growing
cultures or not. This suggests a plate-level growth-limiting
effect. This could conceivably be an experimental limitation such as
the drying out of an agar plate over time. However, comparison of the
striped and filled data, shows that cultures tend to grow larger when
neighbours are removed and this suggests a direct interaction between
cultures. The strongest candidates are competition for nutrients and
growth limiting signalling such as ethanol poisoning. It is possible
that other growth limiting effects may exist and could confound any
attempt to fit a model which accounts for just one of these. It makes
sense to investigate each likely effect in turn to determine its
contribution and to start by validating the independent limit.

We have only studied data where cultures are grown in an array on
solid agar where we cannot validate the independent limit. In this
limit, our model says that nutrients can only be converted to cells
and all cultures starting with the same amount of nutrients will reach
the same final cell density. This ignores metabolism and efficiency of
converting nutrients to cells, which may differ between strains. Cell
arrest could also limit growth. If present, differences in such
effects could account entirely for differences in final cell
density. However, they are unlikely to be the only effect, because
this would not lead to the observed characteristic endpoint in
growth. Using one-culture spot tests (in a perti-dish on ager?) or
liquid cultures we can grow cultures independently and validate the
independent limit. A current issue with methods for estimating
fitness, is that identical strains grow differently on agar or in
liquid culture leading to different fitness rankings (cite). This
problem need not affect our validation as we can simply define a
culture to have different parameters for growth in either medium. A
greater difference may be caused by the dimensionality of the
environment. Mass action kinetics is derived for reactions in a
three-dimentsional (gas or fluid?) (Guldberg and Waage C.M. Guldberg
and P. Waage, Studies Concerning Affinity, C. M. Forhandlinger:
Videnskabs-Selskabet i Christiana (1864), 35) and this approximation
is more valid for liquid cultures than for cultures spotted onto a
surface. I suggest to study first the simpler case of liquid cultures
and later see if the model holds for solid agar. If it does not, it
may be necessary to use a fractal kinetics model or consider a more
detailed model of diffusion of nutrients in agar (if the reaction is
diffusion limited).

// Model equations for metabolism.//


(experiments could be designed to study variation in timescales over
regions of a plate)

% also talk about computational limitations of more complicated
% modelling approaches. Can do this when describing the motivation for
% the fought for model.

\subsection{\thesubsection~Subsection}

%%% Local Variables:
%%% mode: latex
%%% TeX-master: "report"
%%% End:
