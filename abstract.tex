\section*{Abstract}
\label{sec:abstract}


\textbf{Motivation:}~Growth rate is a major component of the
evolutionary fitness of microbial organisms; when nutrients are
plentiful, faster-growing strains come to dominate populations,
whereas slower-growing strains are wiped out. Maintaining a high
growth rate is, therefore, a priority for cell lineages, making growth
rate an excellent surrogate for cell health. When measured in
different genetic backgrounds or environments, cell health can inform
about genetic interaction and drug sensitivity. In the high-throughput
procedures QFA and SGA, microbial cultures are inoculated in an array
on solid agar, and growth is measured to obtain quantitative estimates
of culture fitness. Neighbouring cultures, which are often different
genetic strains, grow at different rates, and consume nutrients at
different rates, creating gradients in nutrient density between
cultures. We believe that diffusion of nutrients between cultures
affects growth. Current analysis does not account for this; instead,
it is assumed that cultures grow independently. I use a network model
of nutrient diffusion and nutrient-dependent growth, to correct for
competition, to try to improve the accuracy and precision of growth
and fitness estimates. I test the model against QFA data from
experiments on \textit{Saccharomyces cerevisiae}. Ultimately, the
model might be used to estimate genetic interaction and drug
sensitivity with higher accuracy and precision.
\\
%Original
% \textbf{Motivation:}~Growth rate is a major component of the
% evolutionary fitness of microbial organisms.~When nutrients are
% plentiful, fast-growing strains come to dominate populations whereas
% slower-growing strains are wiped out. Maintaining growth rate is
% therefore a priority for cell lineages and this makes growth rate an
% excellent surrogate for the health of cells. Measuring the health of
% cells grown in different genetic backgrounds or environments can
% inform about genetic interaction and drug sensitivity. In
% high-throughput procedures such as QFA and SGA, arrays of microbial
% cultures are grown on solid agar plates and quantitative fitness
% estimates are determined from growth measurements. Diffusion of
% nutrients along gradients in nutrient density arising between fast-
% and slow-growing neighbours is likely to affect growth rate and
% fitness estimates. However, current analyses assume that cultures grow
% independently. I study data from QFA experiments growing
% \textit{Saccharomyces cerevisiae} to test a mass action kinetic model
% of nutrient dependent growth and diffusion across a network. I try to
% correct for competition to provide more accurate and precise fitness
% estimates to allow more accurate and precise genetic interaction and
% drug screens.
% \\
\textbf{Results:}~I fit the competition model to a plate from a
previous QFA publication. The plate is inoculated with dilute
cultures; strains have deletions in genes relevant to telomere
function. The competition model fits timecourses with similar
closeness to the previous model. It does so using far fewer parameters
(387 vs 1152). The fit is much closer for at least some fast growing
strains. Fitness estimates are less precise for the fastest growing
strains, but more precise for the majority of strains (36 out of
50). Fitness rankings agree in the positions of the fastest and
slowest growing strains, but disagree in the middle positions. The
positions of the fastest and slowest growing strains also agree with
independent spot tests in published papers. Independent data is not
available to resolve the disagreement in middle rankings. In
cross-plate calibration and validation using QFA plates with a higher
inoculum density, the model overcorrects for competition to a similar
degree as previous models undercorrect. A different method of fitting
is required to find globally optimal solutions.
\\
% Original
% \textbf{Results:}~Using far fewer parameters (387 vs 1152), the
% proposed model fits timecourses from a 384-format QFA plate inoculated
% with dilute cultures with a similar accuracy to a current QFA
% model. Accuracy is improved for fast growing strains. Fitness rankings
% of strains on the same plate agree with a previous QFA study and
% independent experiments in the positions of the fastest and slowest
% growing strains, but disagree for medium growing strains. Estimates
% are less presise for the fastest growing strains, but more precise for
% 36 out of 50 strains. In cross plate callibration and validation,
% using QFA plates with a higher inoculum density, the model
% overcorrects for competition effects to a similar degree that current
% models undercorrect. A different method of fitting is required to find
% globally optimal solutions.
%\\
\textbf{Availability and Implementation:}~CANS, a Python package
developed for the analysis in this paper, is freely available at
\href{https://github.com/lwlss/CANS}{https://github.com/lwlss/CANS}.
%%% Local Variables:
%%% mode: latex
%%% TeX-master: "report"
%%% End:
