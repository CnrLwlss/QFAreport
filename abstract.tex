\section*{Abstract}
\label{sec:abstract}

\textbf{Motivation:}~Growth rate is a major component of the
evolutionary fitness of microbial organisms.~When nutrients are
plentiful, fast-growing strains come to dominate populations whereas
slower-growing strains are wiped out. This makes growth rate an
excellent surrogate for the health of cells. Measuring the health of
cells grown in different genetic backgrounds or environments and can
inform about genetic interaction and drug sensitivity. In
high-throughput procedures such as QFA and SGA, arrays of microbial
cultures are grown on solid agar plates and quantitative fitness
estimates are determined from growth measurements. Diffusion of
nutrients along gradients in nutrient density arising between fast-
and slow-growing neighbours is likely to be affecting growth rate and
fitness estimates. However, current analyses assume that cultures grow
independently. We study data from a QFA experiments growing
\textit{Saccharomyces cerevisiae} to test a model of nutrient
dependent growth and diffusion. We try to correct for competition to
provide more accurate and reliable fitness estimates.
\\
\textbf{Results:}~Don't know what to say yet.\\
\textbf{Availability and Implementation:}~CANS, a Python package
developed for the analysis in this paper, is freely available from
github.
\\
\textbf{Contact:}~daniel.boocock@protonmail.ch\\

%%% Local Variables:
%%% mode: latex
%%% TeX-master: "report"
%%% End:
