\section{Aims}
\label{sec:aims}
Quantitative fitness analysis (QFA) is a method for inferring the
fitness of microbial cultures from their growth curves and can be used
to conduct genome-wide screening for genetic interactions or drug
responses (see
\citet{Addinall2008,Addinall2011,Lawless2010,Banks2012,Andrew2013}). In
QFA, growth curves are automatically captured for arrays of typically
384 cultures grown on solid agar, with each culture containing an
individual genetic strain. Plates can contain any combination of
different strains or repeats. % Microbial organisms have evolved
% redundancies such that very few single gene knockouts are lethal \cite. due
% to redundany microbial fitness is a useful phenotype for measuring
% genetic interaction \cite{Costanzo2010}. If a greater than expected
% loss of fitness occurs in a double mutant knockout this indicates that
% the genes are functionally related because very few genes are
% essential so microbial allowing in the same pathway are expected
Usually QFA cultures are assumed to grow independently
(e.g. \citet{Addinall2011}). We aim to test the validity of this
assumption in a range of different QFA experiments using a new model
of population growth which includes competition and signalling between
cultures. We expect that accounting explicitly for competition and
signalling in analysis of QFA data will increase the reproducibility
of fitness estimates and ultimately increase the statistical power of
QFA in screens for genetic interactions and drug responses.

An analysis by \citet{Baryshnikova2010}, of data obtained using
1536-pin synthetic genetic array (SGA), an alternative procedure also
using an array of cultures on solid agar, attempts to normalise for
systematic variation in growth observations, which may include
variation from competition and signalling effects, using statistical
techniques aimed at improving the correlation between repeats of
identical mutant strains at different locations. By accounting for
competition and signalling mechanistically, we hope to learn more
about the sources of systematic variation in both QFA and SGA
experiments, and to develop analyses and experimental designs to deal
with them. We then aim to determine whether accounting for competition
and signalling will improve variability and rank order of fitness
estimates using unpublished QFA data for the model organism
\textit{Saccharomyces cerevisiae}. We will package all models and
analysis tools so that they may be used in future studies or to
reanalyse data from past studies.

\subsection{Hypotheses}
This project aims to test the following hypotheses:
\begin{itemize}
\item Growth of cultures in QFA and SGA is affected by competition for
  nutrients and/or signalling.
\item By accounting mechanistically for competition and signalling
  effects, it will be possible to design analyses and experiments
  which minimise them.
\item Explicitly accounting for competition and signalling effects in
  data analysis will improve the reproducibility of fitness estimates.
\end{itemize}

%%% Local Variables:
%%% mode: latex
%%% TeX-master: "proposal"
%%% End:
